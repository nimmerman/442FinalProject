\documentclass[10pt,twocolumn,letterpaper]{article}

\usepackage{cvpr}
\usepackage{times}
\usepackage{epsfig}
\usepackage{graphicx}
\usepackage{amsmath}
\usepackage{amssymb}
\usepackage{enumerate}
\usepackage{graphicx}

% Include other packages here, before hyperref.

% If you comment hyperref and then uncomment it, you should delete
% egpaper.aux before re-running latex.  (Or just hit 'q' on the first latex
% run, let it finish, and you should be clear).
\usepackage[breaklinks=true,bookmarks=false]{hyperref}

\cvprfinalcopy % *** Uncomment this line for the final submission

\def\cvprPaperID{****} % *** Enter the CVPR Paper ID here
\def\httilde{\mbox{\tt\raisebox{-.5ex}{\symbol{126}}}}

% Pages are numbered in submission mode, and unnumbered in camera-ready
%\ifcvprfinal\pagestyle{empty}\fi
\setcounter{page}{1}
\begin{document}

%%%%%%%%% TITLE
\title{EECS 442 Computer Vision: Final Project Final Report}

\author{Nathan Immerman\\
College of Engineering, University of Michigan\\
Ann Arbor, Michigan\\
{\tt\small immerman@umich.edu}
% For a paper whose authors are all at the same institution,
% omit the following lines up until the closing ``}''.
% Additional authors and addresses can be added with ``\and'',
% just like the second author.
% To save space, use either the email address or home page, not both
\and
Alexander Chocron\\
College of Engineering, University of Michigan\\
Ann Arbor, Michigan\\
{\tt\small achocron@umich.edu}
}

\maketitle
%\thispagestyle{empty}

%%%%%%%%% BODY TEXT
\section{Introduction}

With the advent of software tools such as Photoshop and Gimp, it is becoming increasingly simple to doctor and create fake images. One method of doctoring images is to create a composite image out of existing source images. Since composite images are difficult to detect for the common person, we hope to create a tool that enables users to determine composite images.

It is often difficult to make the light direction throughout the entirety of composite images consistent. This fact can be leveraged to detect whether even well-stitched images are fake. By analyzing the light direction of different surfaces within the image, one can detect whether or not the two surfaces came from the same original image. An algorithm for detecting such inconsistencies has been outlined in Johnson and Farid's paper, \emph{Exposing digital forgeries by detecting inconsistencies in lighting}. We shall attempt our own implementation of this algorithm and measure our implementation based on accuracy.
%------------------------------------------------------------------------
\section{Approach}

The equation that we are making use of to estimate the light directions is \[I(x,y) = R\times (\vec{N}(x,y)\cdot \vec{L}) + A\]
where I(x,y) is the intensity at the point (x,y), R is the reflectance term (a constant), and N(x,y) is the unit vector normal to the boundary at the point (x,y). This equation assumes an infinite light source.

A limitation of this equation is that it requires that the reflectance of the surface is known and that it is constant. By assuming that the reflectance equals 1, a vector for the light direction can be obtained with an unknown scale factor. To solve the problem of constant reflectance over an object boundary, we split the user entered boundaries into n partitions, and assume that each of these partitions has constant reflectance. In our implementation, we use n = 8.

The reason we must estimate points that lie on a (occluding) boundary is that we are only able to estimate the normal vectors at the occluding boundary for a given object. We know that the z-component of the normal vector at the occluding boundary of an object is 0. This makes it much easier to estimate the x and y components. For each partition, we fit a quadratic curve using three points that the user enters. The user enters a fourth point for each patch as an indicator to the general direction of the normal vectors. We then use mathematical techniques to estimate the normal vector of the quadratic curve at the x coordinate of each point in the given partition.

Further, we need to estimate the intensities at the occluding boundary because the pixels are not in the image. We fit an exponential curve to the points along the normal vector on the object, and extrapolate to estimate the intensity along the boundary.

We cannot take a direct approach to solving for the light directions; we have an unknown ambient term and several points in each partition segment. To overcome this, the paper reformulates the problem in such a way that it is possible to apply least squares estimation to solve for the unknowns. After this is complete, we estimate the overall light direction on the object by averaging the light direction of each segment in the partition.


%------------------------------------------------------------------------
\section{Implementation}

%------------------------------------------------------------------------
\section{Experiments}

%------------------------------------------------------------------------
\section{Conclusion}

%--------------------------------	----------------------------------------
\section{References}

[1] M. K. Johnson and H. Farid. Exposing digital forgeries by detecting inconsistencies in lighting. \emph{In Proceedings of the 7th workshop on Multimedia and security}, pages 1-10, 2005.


\end{document}































