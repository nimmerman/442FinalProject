\documentclass[10pt,twocolumn,letterpaper]{article}

\usepackage{cvpr}
\usepackage{times}
\usepackage{epsfig}
\usepackage{graphicx}
\usepackage{amsmath}
\usepackage{amssymb}
\usepackage{enumerate}
\usepackage{graphicx}

% Include other packages here, before hyperref.

% If you comment hyperref and then uncomment it, you should delete
% egpaper.aux before re-running latex.  (Or just hit 'q' on the first latex
% run, let it finish, and you should be clear).
\usepackage[breaklinks=true,bookmarks=false]{hyperref}

\cvprfinalcopy % *** Uncomment this line for the final submission

\def\cvprPaperID{****} % *** Enter the CVPR Paper ID here
\def\httilde{\mbox{\tt\raisebox{-.5ex}{\symbol{126}}}}

% Pages are numbered in submission mode, and unnumbered in camera-ready
%\ifcvprfinal\pagestyle{empty}\fi
\setcounter{page}{1}
\begin{document}

%%%%%%%%% TITLE
\title{EECS 442 Computer Vision: Final Project Final Report}

\author{Nathan Immerman\\
College of Engineering, University of Michigan\\
Ann Arbor, Michigan\\
{\tt\small immerman@umich.edu}
% For a paper whose authors are all at the same institution,
% omit the following lines up until the closing ``}''.
% Additional authors and addresses can be added with ``\and'',
% just like the second author.
% To save space, use either the email address or home page, not both
\and
Alexander Chocron\\
College of Engineering, University of Michigan\\
Ann Arbor, Michigan\\
{\tt\small achocron@umich.edu}
}

\maketitle
%\thispagestyle{empty}

%%%%%%%%% BODY TEXT
\section{Introduction}

With the advent of software tools such as Photoshop and Gimp, it is becoming increasingly simple to doctor and create fake images. One method of doctoring images is to create a composite image out of existing source images. Since composite images are difficult to detect for the common person, we hope to create a tool that enables users to determine composite images.
%------------------------------------------------------------------------
\section{Approach}

It is often difficult to make the light direction throughout the entirety of composite images consistent. This fact can be leveraged to detect whether even well-stitched images are fake. By analyzing the light direction of different surfaces within the image, one can detect whether or not the two surfaces came from the same original image. An algorithm for detecting such inconsistencies has been outlined in Johnson and Farid's paper, \emph{Exposing digital forgeries by detecting inconsistencies in lighting}. We shall attempt our own implementation of this algorithm and measure our implementation based on accuracy.


%------------------------------------------------------------------------
\section{Implementation}

%------------------------------------------------------------------------
\section{Experiments}

%------------------------------------------------------------------------
\section{Conclusion}

%--------------------------------	----------------------------------------
\section{References}

[1] M. K. Johnson and H. Farid. Exposing digital forgeries by detecting inconsistencies in lighting. \emph{In Proceedings of the 7th workshop on Multimedia and security}, pages 1-10, 2005.


\end{document}































