\documentclass[10pt,twocolumn,letterpaper]{article}

\usepackage{cvpr}
\usepackage{times}
\usepackage{epsfig}
\usepackage{graphicx}
\usepackage{amsmath}
\usepackage{amssymb}
\usepackage{enumerate}
\usepackage{graphicx}

% Include other packages here, before hyperref.

% If you comment hyperref and then uncomment it, you should delete
% egpaper.aux before re-running latex.  (Or just hit 'q' on the first latex
% run, let it finish, and you should be clear).
\usepackage[breaklinks=true,bookmarks=false]{hyperref}

\cvprfinalcopy % *** Uncomment this line for the final submission

\def\cvprPaperID{****} % *** Enter the CVPR Paper ID here
\def\httilde{\mbox{\tt\raisebox{-.5ex}{\symbol{126}}}}

% Pages are numbered in submission mode, and unnumbered in camera-ready
%\ifcvprfinal\pagestyle{empty}\fi
\setcounter{page}{1}
\begin{document}

%%%%%%%%% TITLE
\title{EECS 442 Computer Vision: Final Project Final Report}

\author{Nathan Immerman\\
College of Engineering, University of Michigan\\
Ann Arbor, Michigan\\
{\tt\small immerman@umich.edu}
% For a paper whose authors are all at the same institution,
% omit the following lines up until the closing ``}''.
% Additional authors and addresses can be added with ``\and'',
% just like the second author.
% To save space, use either the email address or home page, not both
\and
Alexander Chocron\\
College of Engineering, University of Michigan\\
Ann Arbor, Michigan\\
{\tt\small achocron@umich.edu}
}

\maketitle
%\thispagestyle{empty}

%%%%%%%%% BODY TEXT
\section{Introduction}


%------------------------------------------------------------------------
\section{Approach}

%------------------------------------------------------------------------
\section{Implementation}

%------------------------------------------------------------------------
\section{Experiments}

%------------------------------------------------------------------------
\section{Conclusion}

%--------------------------------	----------------------------------------
\section{References}

[1] M. K. Johnson and H. Farid. Exposing digital forgeries by detecting inconsistencies in lighting. \emph{In Proceedings of the 7th workshop on Multimedia and security}, pages 1-10, 2005.


\end{document}































